% Aditya Krishna, Amatullah Mir, Edwin Santos, and Zixuan Zhang (Group 6)


\documentclass[12pt]{article}


%----------Packages----------
\usepackage{amsmath}
\usepackage{amssymb}
\usepackage{amsthm}
%\usepackage{amsrefs}
\usepackage{dsfont}
\usepackage{mathrsfs}
\usepackage{stmaryrd}
\usepackage[all]{xy}https://www.overleaf.com/project/5f9cd81364a7ee00010b72b4
\usepackage[mathcal]{eucal}
\usepackage{verbatim}  %%includes comment environment
\usepackage{fullpage}  %%smaller margins
\usepackage{hyperref}
\usepackage{setspace}
\onehalfspacing
%----------Commands----------

%%penalizes orphans
\clubpenalty=9999
\widowpenalty=9999

%% bold math capitals
\newcommand{\bA}{\mathbf{A}}
\newcommand{\bB}{\mathbf{B}}
\newcommand{\bC}{\mathbf{C}}
\newcommand{\bD}{\mathbf{D}}
\newcommand{\bE}{\mathbf{E}}
\newcommand{\bF}{\mathbf{F}}
\newcommand{\bG}{\mathbf{G}}
\newcommand{\bH}{\mathbf{H}}
\newcommand{\bI}{\mathbf{I}}
\newcommand{\bJ}{\mathbf{J}}
\newcommand{\bK}{\mathbf{K}}
\newcommand{\bL}{\mathbf{L}}
\newcommand{\bM}{\mathbf{M}}
\newcommand{\bN}{\mathbf{N}}
\newcommand{\bO}{\mathbf{O}}
\newcommand{\bP}{\mathbf{P}}
\newcommand{\bQ}{\mathbf{Q}}
\newcommand{\bR}{\mathbf{R}}
\newcommand{\bS}{\mathbf{S}}
\newcommand{\bT}{\mathbf{T}}
\newcommand{\bU}{\mathbf{U}}
\newcommand{\bV}{\mathbf{V}}
\newcommand{\bW}{\mathbf{W}}
\newcommand{\bX}{\mathbf{X}}
\newcommand{\bY}{\mathbf{Y}}
\newcommand{\bZ}{\mathbf{Z}}

%% blackboard bold math capitals
\newcommand{\bbA}{\mathbb{A}}
\newcommand{\bbB}{\mathbb{B}}
\newcommand{\bbC}{\mathbb{C}}
\newcommand{\bbD}{\mathbb{D}}
\newcommand{\bbE}{\mathbb{E}}
\newcommand{\bbF}{\mathbb{F}}
\newcommand{\bbG}{\mathbb{G}}
\newcommand{\bbH}{\mathbb{H}}
\newcommand{\bbI}{\mathbb{I}}
\newcommand{\bbJ}{\mathbb{J}}
\newcommand{\bbK}{\mathbb{K}}
\newcommand{\bbL}{\mathbb{L}}
\newcommand{\bbM}{\mathbb{M}}
\newcommand{\bbN}{\mathbb{N}}
\newcommand{\bbO}{\mathbb{O}}
\newcommand{\bbP}{\mathbb{P}}
\newcommand{\bbQ}{\mathbb{Q}}
\newcommand{\bbR}{\mathbb{R}}
\newcommand{\bbS}{\mathbb{S}}
\newcommand{\bbT}{\mathbb{T}}
\newcommand{\bbU}{\mathbb{U}}
\newcommand{\bbV}{\mathbb{V}}
\newcommand{\bbW}{\mathbb{W}}
\newcommand{\bbX}{\mathbb{X}}
\newcommand{\bbY}{\mathbb{Y}}
\newcommand{\bbZ}{\mathbb{Z}}

%% script math capitals
\newcommand{\sA}{\mathscr{A}}
\newcommand{\sB}{\mathscr{B}}
\newcommand{\sC}{\mathscr{C}}
\newcommand{\sD}{\mathscr{D}}
\newcommand{\sE}{\mathscr{E}}
\newcommand{\sF}{\mathscr{F}}
\newcommand{\sG}{\mathscr{G}}
\newcommand{\sH}{\mathscr{H}}
\newcommand{\sI}{\mathscr{I}}
\newcommand{\sJ}{\mathscr{J}}
\newcommand{\sK}{\mathscr{K}}
\newcommand{\sL}{\mathscr{L}}
\newcommand{\sM}{\mathscr{M}}
\newcommand{\sN}{\mathscr{N}}
\newcommand{\sO}{\mathscr{O}}
\newcommand{\sP}{\mathscr{P}}
\newcommand{\sQ}{\mathscr{Q}}
\newcommand{\sR}{\mathscr{R}}
\newcommand{\sS}{\mathscr{S}}
\newcommand{\sT}{\mathscr{T}}
\newcommand{\sU}{\mathscr{U}}
\newcommand{\sV}{\mathscr{V}}
\newcommand{\sW}{\mathscr{W}}
\newcommand{\sX}{\mathscr{X}}
\newcommand{\sY}{\mathscr{Y}}
\newcommand{\sZ}{\mathscr{Z}}

\renewcommand{\phi}{\varphi}
%\renewcommand{\emptyset}{\O}

\providecommand{\abs}[1]{\lvert #1 \rvert}
\providecommand{\norm}[1]{\lVert #1 \rVert}
\providecommand{\x}{\times}
\providecommand{\ar}{\rightarrow}
\providecommand{\arr}{\longrightarrow}


%----------Theorems----------

\newtheorem{theorem}{Theorem}[section]
\newtheorem{proposition}[theorem]{Proposition}
\newtheorem{lemma}[theorem]{Lemma}
\newtheorem{corollary}[theorem]{Corollary}

\theoremstyle{definition}
\newtheorem{definition}[theorem]{Definition}
\newtheorem{nondefinition}[theorem]{Non-Definition}
\newtheorem{exercise}[theorem]{Exercise}

\numberwithin{equation}{subsection}



\begin{document}

\pagestyle{plain}


%%---  sheet number for theorem counter
%\setcounter{section}{1}   

\begin{center}
{\large Math 15910 Assignment 4} \\ 
\vspace{.2in}  
\end{center}



\section{Exercise 1}
Prove that if x $>$ -1 then, for every $n \in \mathbb{N}$, we have $(1+x)^n \geq 1 + nx$

\begin{proof}
If x $>$ -1, $\forall n \in \mathbb{N}$, then P(n):= $(1+x)^n \geq 1 + nx$.

1) P(1): $(1+x)^n \geq 1 + nx$ where n = 1
\\ $(1+x)^1 \geq 1+1*x$
\\ $1+x \geq 1+x$

2) $\forall k \in \mathbb{N}$, P(k) $\implies$ P(k+1)
\\ If P(k) = $(1+x)^k \geq 1 + kx$
\\ Then P(k+1) = $(1+x)^{k+1} \geq 1 + (k+1)x$
\\ (1+x)*P(k) = $(1+x)^1*(1+x)^k \geq (1+kx)*(1+x)$
\\  = $(1+x)^{k+1} \geq 1 + kx + x + kx^2$
\\  = $(1+x)^{k+1} \geq 1 + (k+1)x + kx^2$
\\ $1 + (k+1)x + kx^2 \geq 1 + (k+1)x$ if $kx^2 \geq 0$
\\ Such that if $(1+x)^{k+1} \geq 1 + (k+1)x + kx^2$
\\ then $(1+x)^{k+1} \geq 1 + (k+1)x + kx^2 \geq 1 + (k+1)x$
\\ and $(1+x)^{k+1} \geq 1 + (k+1)x$

Therefore, P(n) is true $\forall n \in \mathbb{N}$
\end{proof}

\section{Exercise 2}
Let a,b $\in \mathbb{R}$
\begin{proof}
\\ (a) If a + b = 0 then b = -a
\\ b = b + 0 (Additive Identity)
\\ b = b + (a + (-a)) (Additive Inverse)
\\ b = (b + a) + (-a) (Additive Associative)
\\ b = 0 + (-a) (Proposition a + b = 0)
\\ b = -a (Additive Identity)
\\ (b) (-1)*a = -a
\\ -a = (-1)*a
\\ -a = (-a*$\frac{1}{a}$)*a (Multiplicative Inverse)
\\ -a = (-a)*($\frac{1}{a}$*a) (Multiplicative Associative)
\\ -a = (-a)*1 (Multiplicative Inverse)
\\ -a = -a (Multiplicative Identity)
\\ (c) (-1)*(-1) = 1
\\ 0 = 0
\\ 0 = 0*0 (Lemma C*0 = 0 where C = 0)
\\ 0 = (1-1)*(1-1) (Additive Inverse)
\\ 0 = 1*1 + 1*(-1) + (-1)*1 + (-1)*(-1) (Distributive)
\\ 0 = 1 + (-1) + (-1) + (-1)*(-1) (Multiplicative Identity)
\\ 0 = 0 + (-1) + (-1)*(-1) (Additive Inverse)
\\ 0 = (-1) + (-1)*(-1) (Additive Identity)
\\ 1 = (-1)*(-1) (Addition)
\\ (d) If a $\neq$ 0 then 1/(-a) = -(1/a)
\\ 1 = (-1)*(-1) (Part 2c)
\\ 1 = (-1)*1*(-1) (Multiplicative Identity)
\\ 1 = (-1)*a*$\frac{1}{a}$*(-1) (Multiplicative Inverse)
\\ 1 = (-a)*$\frac{1}{a}$*(-1) (Part 2b)
\\ $\frac{1}{(-a)}$ = $\frac{1}{a}$*(-1) (Division)
\\ $\frac{1}{(-a)}$ = -($\frac{1}{a}$) 
\\ (e) If a*a = a then a =0 or a = 1
\\ Part 1
\\ a = a*1 (Multiplicative Identity)
\\ a = a*(a*$\frac{1}{a}$) (Multiplicative Inverse)
\\ a = (a*a)*($\frac{1}{a}$) (Multiplicative Associative)
\\ a = (a)*($\frac{1}{a}$) (Proposition a*a = a)
\\ a = 1 (Multiplicative Inverse)
\\ Part 2
\\ a*a = a
\\ a*a - a = 0 (Subtraction)
\\ a(a - 1) = 0 (Distributive)
\\ a = 0*$\frac{1}{a-1}$ (Division)
\\ a = 0 (Lemma C*0 = 0)
\end{proof}

\section{Exercise 3.1.3}
\textbf{Let a be a positive rational number. \\
Let A = $\{x \in\ \mathbb Q\ |\x^2 < a\}$. Show that A is bounded in $\mathbb Q$\.}

\begin{definition}
Let F be an ordered field. Let A be a non-empty set of F. We say that A is bounded above if there is an element M $\in$ F with the property that if $x \in $ A, then $x \leq$ M. We say that A is bounded below if there is an element m $\in$ F such that if x $\in$ A, then m $\leq$ x. (m is a lower bound for A and M is the upper bound of A. We say that if A has an upper bound and a lower bound, then A is bounded.
\end{definition}
If $x^2 < a$, the square root of x yields both the negative and positive square root of a. 
\begin{center}
\\
$x^2 < a$,
\\
$x < \sqrt{a}$ and $x > \sqrt{a}$,
\\
$-\sqrt{a} < x < \sqrt{a}$,
\end{center}
$-\sqrt{a}$ is the lower bound of A, and $\sqrt{a}$ is the upper bound of A. Using Definition 3.1, because A has an upper and lower bound, it is bounded in $\mathbb Q$.



\section{Exercise 3.1.7}
\textbf{Find the least upper bound in $\mathbb R$ of the set A in Exercise 3.1.3.}
\\
$\{x \in\ \mathbb Q\ |\x^2 < a\}$
\\
We note that every element of $\mathbb{Q}$ is less than $\sqrt{a}$ since $x < \sqrt{a}$.
\\
Assume that $\sqrt{a}$ is not the least upper bound. There must exist an $\epsilon > 0$ such that $\sqrt{a} - \epsilon$ is also an upper bound. We now claim that there is a positive rational number $a$ such that $\sqrt{a} - \epsilon < \sqrt{a}$. As $a$ is positive this always holds (there is no $\epsilon$ greater than 0 that when subtracted makes a positive integer more than itself). This means that $\sqrt{a}$ is the least upper bound. 
\section{Exercise 3.1.14}
  1)we define x=lub($A\cup B$)
\\then x$\geq e$ for every element e in $A\cup B$, and therefore x is greater than every element in A and in B. 
\\then x$\geq lub(A)$, x$\geq lub(B)$
\\then x $\geq$ max(lub(A), lub(B)).
\\inverse, we define x=max(lub(A), lub(B)).
\\then $x\geq lub(A), x\geq lub(B) $. then x is greater than all elements in A and in B, x $\geq lub(A\cup B)$.
\\To conclude, we got lub($A\cup B$)=max(lub(A), lub(B)). 

\\2)$lub(A) \geq a, lub(b)\geq b,$ then $lub(a)+lub(b)\geq a+b.$ 
\\since lub(A)+lub(B) is an upper bound for a+b, $lub(A)+lub(B)\geq lub(A+B).$
\\inverse, $lub(A+B)\geq a+b$ for every element a in A and b in B. then $lub(A+B)-a \geq b$, then lub(A+B)-a is an upper bound for b and then $lub(A+B)-a \geq lub(B)$.
\\then we arrange $lub(A+B)-lub(B) \geq a$, and then $lub(A+B)-lub(B) \geq lub(a)$, $lub(A+B) \geq lub(a)+lub(B)$.
\\To conclude, $lub(A+B)= lub(a)+lub(B)$
\\
\\3) $lub(A) \geq a, lub(b)\geq b,$ then $lub(a)\cdot lub(b)\geq ab$.
\\ $lub(A \cdot B) \geq ab$ for every element a in A and b in B. It follows that $lub(A \cdot B)/a \geq b$. It follows that $lub(A \cdot B)/a \geq lub(b)$ and so $lub(A \cdot B)/lub(b) \geq a$.This also means that $lub(A \cdot B)/b \geq a$. It follows that $lub(A \cdot B)/b \geq lub(a)$ and so $lub(A \cdot B)/lub(a) \geq b$. In conclusion $lub(A \cdot B)= lub(a)lub(B)$
\\
\\4) glb($A\cupB$) = min$\{$glb($A$),glb($A$)$\}$
\\ If $A+B = \{a + b| a \in A, b \in B\}$ then glb($A+B$) = glb($A$) + glb($B$)
\\ If the elements of $A$ and $B$ are positive and $A\cdot B = {ab | a \in A, b \in B}$, then glb($A \cdot B$) = glb($A$)glb($B$)

\section{Exercise 3.2.9}
1) we define the irrational number as k and the two rational numbers as $p_1/q_1, p_2/q_2$. for $p_i,q_i\in {\mathbb Z}$ and not equal to 0 
\\ then we assume that k $\times p_1/q_1=p_2/q_2$.
\\Since $p_1/q_1, p_2/q_2\neq 0$, k=$p_1*p_2/q_1*q_2$
\\then we say since $p_i,q_i\in {\mathbb Z}$, $p_1*p_2,  q_1*q_2\in {\mathbb Z}$. Then k can be expressed in the form of p/q, where $p,q\in {\mathbb Z}$. However, irrational cannot be expressed in such form, thus contradiction is revealed. 
\\We prove that irrational times a non-zero rational cannot be a rational.
\\
2) We show two examples:
\\ $\sqrt{2}*\sqrt{2}=2$ is irrational times irrational equals rational.
\\ $\sqrt{2}*\sqrt{3}=\sqrt{6}$ is irrational  times irrational equals irrational.
\\Then we show that since rational and the irrational combine is the set of real number R, irrational is a subset of R. 
\\Since R is closed under multiplication, irrational multiplication can have result only in R, irrational and rational. Since we show that both is possible, we prove it.


\section{Bonus}
We show in this part that p($\mathbb{R}$) is strictly bigger than $\mathbb{R}$ in cardinality through showing $\mathbb{R}$ to p($\mathbb{R}$) exists injection but not surjection.
\\for injection, for any element $w\in \mathbb{R} $, w represent $\{w\}$ in p($\mathbb{R}$).
\\for surjection, we assume there existed surjection. Then p($\mathbb{R}$) can be expressed as (f($w_1$),f($w_2$),...) for all $w_i \in \mathbb{R}$. Therefore we construct a set B=$\{x\in \mathbb{R} | x\notin f(x) \}$. And therefore B must not be in the range of f. 
Since $x\in \mathbb{R}$, then B is in p($\mathbb{R}$). Since we assume surjection, B must can be expressed by f(x) for x in $\mathbb{R}$. 
\\Therefore, contradiction, $\mathbb{R}$ to p($\mathbb{R}$) is injective but not surjective. 
\\ Therefore $|\mathbb{R}|$ is strictly smaller than $|p(\mathbb{R})|$.
\end{document}