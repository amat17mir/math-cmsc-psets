% Aditya Krishna, Amatullah Mir, Edwin Santos, Melanie Zhang, and Zixuan Zhang (Group 6)

%LATEX TIPS (I'll think of more and feel free to add as you work thru this)
% \\ for new line
% enclose math in dollar signs, i.e. $[type math]$
% https://oeis.org/wiki/List_of_LaTeX_mathematical_symbols  for symbols
% \ for space, \enspace for bigger space \quad for even bigger
% {^{\mathrm{c}}} is the only way I know to type the complement symbol :(
% more in Kevin's doc on canvas 

\documentclass[12pt]{article}


%----------Packages----------
\usepackage{amsmath}
\usepackage{amssymb}
\usepackage{amsthm}
%\usepackage{amsrefs}
\usepackage{dsfont}
\usepackage{mathrsfs}
\usepackage{stmaryrd}
\usepackage[all]{xy}
\usepackage[mathcal]{eucal}
\usepackage{verbatim}  %%includes comment environment
\usepackage{fullpage}  %%smaller margins
\usepackage{hyperref}
\usepackage{setspace}
\onehalfspacing
%----------Commands----------

%%penalizes orphans
\clubpenalty=9999
\widowpenalty=9999

%% bold math capitals
\newcommand{\bA}{\mathbf{A}}
\newcommand{\bB}{\mathbf{B}}
\newcommand{\bC}{\mathbf{C}}
\newcommand{\bD}{\mathbf{D}}
\newcommand{\bE}{\mathbf{E}}
\newcommand{\bF}{\mathbf{F}}
\newcommand{\bG}{\mathbf{G}}
\newcommand{\bH}{\mathbf{H}}
\newcommand{\bI}{\mathbf{I}}
\newcommand{\bJ}{\mathbf{J}}
\newcommand{\bK}{\mathbf{K}}
\newcommand{\bL}{\mathbf{L}}
\newcommand{\bM}{\mathbf{M}}
\newcommand{\bN}{\mathbf{N}}
\newcommand{\bO}{\mathbf{O}}
\newcommand{\bP}{\mathbf{P}}
\newcommand{\bQ}{\mathbf{Q}}
\newcommand{\bR}{\mathbf{R}}
\newcommand{\bS}{\mathbf{S}}
\newcommand{\bT}{\mathbf{T}}
\newcommand{\bU}{\mathbf{U}}
\newcommand{\bV}{\mathbf{V}}
\newcommand{\bW}{\mathbf{W}}
\newcommand{\bX}{\mathbf{X}}
\newcommand{\bY}{\mathbf{Y}}
\newcommand{\bZ}{\mathbf{Z}}

%% blackboard bold math capitals
\newcommand{\bbA}{\mathbb{A}}
\newcommand{\bbB}{\mathbb{B}}
\newcommand{\bbC}{\mathbb{C}}
\newcommand{\bbD}{\mathbb{D}}
\newcommand{\bbE}{\mathbb{E}}
\newcommand{\bbF}{\mathbb{F}}
\newcommand{\bbG}{\mathbb{G}}
\newcommand{\bbH}{\mathbb{H}}
\newcommand{\bbI}{\mathbb{I}}
\newcommand{\bbJ}{\mathbb{J}}
\newcommand{\bbK}{\mathbb{K}}
\newcommand{\bbL}{\mathbb{L}}
\newcommand{\bbM}{\mathbb{M}}
\newcommand{\bbN}{\mathbb{N}}
\newcommand{\bbO}{\mathbb{O}}
\newcommand{\bbP}{\mathbb{P}}
\newcommand{\bbQ}{\mathbb{Q}}
\newcommand{\bbR}{\mathbb{R}}
\newcommand{\bbS}{\mathbb{S}}
\newcommand{\bbT}{\mathbb{T}}
\newcommand{\bbU}{\mathbb{U}}
\newcommand{\bbV}{\mathbb{V}}
\newcommand{\bbW}{\mathbb{W}}
\newcommand{\bbX}{\mathbb{X}}
\newcommand{\bbY}{\mathbb{Y}}
\newcommand{\bbZ}{\mathbb{Z}}

%% script math capitals
\newcommand{\sA}{\mathscr{A}}
\newcommand{\sB}{\mathscr{B}}
\newcommand{\sC}{\mathscr{C}}
\newcommand{\sD}{\mathscr{D}}
\newcommand{\sE}{\mathscr{E}}
\newcommand{\sF}{\mathscr{F}}
\newcommand{\sG}{\mathscr{G}}
\newcommand{\sH}{\mathscr{H}}
\newcommand{\sI}{\mathscr{I}}
\newcommand{\sJ}{\mathscr{J}}
\newcommand{\sK}{\mathscr{K}}
\newcommand{\sL}{\mathscr{L}}
\newcommand{\sM}{\mathscr{M}}
\newcommand{\sN}{\mathscr{N}}
\newcommand{\sO}{\mathscr{O}}
\newcommand{\sP}{\mathscr{P}}
\newcommand{\sQ}{\mathscr{Q}}
\newcommand{\sR}{\mathscr{R}}
\newcommand{\sS}{\mathscr{S}}
\newcommand{\sT}{\mathscr{T}}
\newcommand{\sU}{\mathscr{U}}
\newcommand{\sV}{\mathscr{V}}
\newcommand{\sW}{\mathscr{W}}
\newcommand{\sX}{\mathscr{X}}
\newcommand{\sY}{\mathscr{Y}}
\newcommand{\sZ}{\mathscr{Z}}

\renewcommand{\phi}{\varphi}
%\renewcommand{\emptyset}{\O}

\providecommand{\abs}[1]{\lvert #1 \rvert}
\providecommand{\norm}[1]{\lVert #1 \rVert}
\providecommand{\x}{\times}
\providecommand{\ar}{\rightarrow}
\providecommand{\arr}{\longrightarrow}


%----------Theorems----------

\newtheorem{theorem}{Theorem}[section]
\newtheorem{proposition}[theorem]{Proposition}
\newtheorem{lemma}[theorem]{Lemma}
\newtheorem{corollary}[theorem]{Corollary}

\theoremstyle{definition}
\newtheorem{definition}[theorem]{Definition}
\newtheorem{nondefinition}[theorem]{Non-Definition}
\newtheorem{exercise}[theorem]{Exercise}

\numberwithin{equation}{subsection}



\begin{document}

\pagestyle{plain}


%%---  sheet number for theorem counter
%\setcounter{section}{1}   

\begin{center}
{\large Math 15910 Assignment 1} \\ 
\vspace{.2in}  
\end{center}


\section{Exercise 1.3.9.}
To show $=$, we show $\subseteq$ and $\supseteq.$
\subsection*{$(ii) \enspace A \cap (B \cap C) = (A \cap B) \cap C $}
\begin{lemma}
$A \cap (B \cap C) \subseteq (A \cap B) \cap C $
\end{lemma}
\begin{proof}
Let $x \in A \cap (B \cap C), \ \text{show} \ x \in (A \cap B) \cap C$ \\
By definition of intersection, $ x \in A \ \text{and}\ x \in (B \cap C)
\implies x \in A \ \text{and}\ x \in B \ \text{and}\  x \in C 
\implies x \in (A \cap B) \ \text{and}\ x \in C 
\implies x \in (A \cap B) \cap C $.
\end{proof}

\begin{lemma}
$(A \cap B) \cap C \subseteq A \cap (B \cap C) $
\end{lemma}
\begin{proof}
Let $x \in (A \cap B) \cap C, \ \text{show} \ x \in A \cap (B \cap C)$ \\
By definition of intersection, $x \in (A \cap B) \ \text{and}\  x \in C
\implies x \in A \ \text{and}\ x \in B \ \text{and}\  x \in C 
\implies x \in A \ \text{and}\ x \in (B \cap C)
\implies x \in A \cap (B \cap C) $.
\end{proof}

Since $A \cap (B \cap C) \subseteq (A \cap B) \cap C $ and $(A \cap B) \cap C \subseteq A \cap (B \cap C) $, we must have $=$. 

\subsection*{$(ix) \enspace A \triangle B = \emptyset \iff  A = B $}
\begin{lemma}
$ A \triangle B \subseteq \emptyset \iff  A = B $
\end{lemma}
\begin{proof}
Suppose $A = B.$ We can begin by assuming $ A \triangle B \nsubseteq \emptyset$.\\
Let $x \in A \triangle B.$ By definition of symmetric difference, $x \in A \triangle B \implies x \in (A \setminus B) \cup (B \setminus A).$ Since $A = B, x \in (A \setminus A) \cup (B \setminus B) \implies x \in (A \cap   {^{\mathrm{c}}}{A}) \ \cup \ x \in (B \cap  {^{\mathrm{c}}}{B}) \implies x \in A $ and $x \notin A$ or $x \in B $ and $x \notin B$. Both these statements are contradictions. Thus, our assumption is false and $ A \triangle B \subseteq \emptyset \iff  A = B. $
\end{proof}

\begin{lemma}
$\emptyset \subseteq  A \triangle B \iff  A = B $
\end{lemma}
\begin{proof}
Suppose $\emptyset \nsubseteq  A \triangle B.$ This suggests $\exists \ x \in \emptyset \ |\  x \notin A \triangle B.$ There cannot exist an object $x$ in the empty set as this contradicts the very definition of the empty set, so our assumption is false and $\emptyset \subseteq  A \triangle B \iff  A = B.$ (This reaffirms the empty set is a subset of every set). 
\end{proof}
Since $ A \triangle B \subseteq \emptyset$ and $\emptyset \subseteq  A \triangle B \iff  A = B $, we must have $=.$

\subsection*{$(x) \enspace A \cap (B \triangle C) = (A \cap B) \triangle (A \cap C)$}
\begin{lemma}
$A \cap (B \triangle C) \subseteq (A \cap B) \triangle (A \cap C)$
\end{lemma}
\begin{proof}
Let $x \in A \cap (B \triangle C),$ show $x \in (A \cap B) \triangle (A \cap C).$\\
By definitions of intersection and symmetric difference, $x \in A \cap (B \triangle C) \implies x \in A$ and $x \in (B \setminus C) \cup (C \setminus B) \implies x \in A$ and $\left [( x \in B \ \text{and} \ x \notin C ) \ \text{or}\ ( x \in C \ \text{and} \ x \notin B) \right].$ \\
Noting we can rewrite $x \notin C$ and $x \notin B$ as complements of sets $C$ and $B$, we get:
\begin{align*}
    x \in A \ \text{and} \left [( x \in B \ \text{and} \ x \in {^{\mathrm{c}}}{C} ) \ \text{or}\ ( x \in C \ \text{and} \ x \in {^{\mathrm{c}}}{B}) \right]
\end{align*}

Seeing as set A has an intersection with $ ( x \in B \ \text{and} \ x \in {^{\mathrm{c}}}{C} ) \ \textbf{or}\ ( x \in C \ \text{and} \ x \in {^{\mathrm{c}}}{B}),$ we use the associative law for intersection (see $(ii)$) for each bracketed set, getting $ (x\in A \ \text{and} \ x \in B \ \text{and} \ x \in {^{\mathrm{c}}}{C} ) \ \text{or}\ (x \in A \ \text{and} \ x \in C \ \text{and} \ x \in {^{\mathrm{c}}}{B}) \implies (x \in A\cap B \cap {^{\mathrm{c}}}{C})\cup (x \in A\cap C \cap {^{\mathrm{c}}}{B}),$ which is equivalent to the redundant expression $(x \in A\cap B \cap A \cap {^{\mathrm{c}}}{C})\cup (x \in A\cap C \cap A \cap {^{\mathrm{c}}}{B}) \implies (x \in A \cap B$ and $x \in A \cap {^{\mathrm{c}}}{C})\cup(x \in A \cap C$ and $x \in A \cap {^{\mathrm{c}}}{B}).$ \\ \\
If $x \in {^{\mathrm{c}}}{C}$, then $x \notin C$ (and similarly for $x \in {^{\mathrm{c}}}{B}$). This gives $(x \in A \cap B$ and $x \notin A \cap C)\cup(x \in A \cap C$ and $x \notin A \cap B).$ By definition of difference, $x \in [(A \cap B)\setminus (A \cap C)] \cup  [(A \cap C)\setminus (A \cap B)].$ By definition of symmetric difference, $x \in (A \cap B) \triangle (A \cap C).$
\end{proof}
\begin{lemma}
$(A \cap B) \triangle (A \cap C) \subseteq A \cap (B \triangle C)$
\end{lemma}
\begin{proof}
Let $x \in (A \cap B) \triangle (A \cap C),$ show $x \in A \cap (B \triangle C).$\\
By definition of symmetric difference, $x \in (A \cap B) \triangle (A \cap C) \implies x \in ((A \cap B) \setminus (A \cap C)) \cup ((A \cap C) \setminus (A \cap B)) \implies [x \in (A \cap B) \ \text{and} \ x \notin (A \cap C)] \ \text{or} \ [x \in (A \cap C) \ \text{and} \ x \notin (A \cap B)].$\\ \\
If $x \in (A \cap B) \ \textbf{and} \ x \notin (A \cap C),$ we know from the first half that $x \in B$ and $x \in A$. Using that information to interpret the second half, it must be that $\ x \notin (A \cap C) \implies x \notin C$. If $x \notin C,$ then $x \in {^{\mathrm{c}}}{C}.$ A similar argument can be made for $x \in (A \cap C) \ \text{and} \ x \notin (A \cap B).$\\\\
Then, our current expression becomes $x \in (A \cap B \cap {^{\mathrm{c}}}{C}) \cup (A \cap C \cap {^{\mathrm{c}}}{B}) \implies (x \in A \ \text{and} \ x \in B \cap {^{\mathrm{c}}}{C}) \cup (x \in A \ \text{and}\ x \in C \cap {^{\mathrm{c}}}{B}) \implies (x \in A \ \text{and} \ x \in B \setminus C) \ \textbf{or} \ (x \in A \ \text{and}\ x \in C \setminus B).$ It is clear that $x \in A$ and either $x \in (B \setminus C)$ or $x \in (C \setminus B).$ Thus, we have $x \in A \cap [(B \setminus C) \cup (C \setminus B)].$ By definition of symmetric difference, $x \in A \cap (B \triangle C).$
\end{proof}
Since $A \cap (B \triangle C) \subseteq (A \cap B) \triangle (A \cap C)$ and $(A \cap B) \triangle (A \cap C) \subseteq A \cap (B \triangle C)$, we must have $=.$ 


\subsection*{$(xi) \enspace A \cup (B \cap C) = (A \cup B) \cap (A \cup C)$}
\begin{lemma}
$A \cup (B \cap C) \subseteq (A \cup B) \cap (A \cup C)$
\end{lemma}
\begin{proof}
Let $x \in A \cup (B \cap C),$ show $x \in (A \cup B) \cap (A \cup C).$\\
By definition of union, $x \in A \cup (B \cap C) \implies x \in A$ or $x \in B \cap C.$\\ Consider the first case: if $x \in A,$ then $x \in A \cup B$ and $x \in A \cup C$ are both true. Then, $x \in (A \cup B) \cap (A \cup C).$ Now, consider the second case: if $x \in B \cap C,$ then $x \in B$ and $x \in C.$ This makes $x \in A \cup B$ and $x \in A \cup C$ true, and by definition of intersection, $x \in (A \cup B) \cap (A \cup C).$
\end{proof}
\begin{lemma}
$(A \cup B) \cap (A \cup C) \subseteq A \cup (B \cap C)$
\end{lemma}
\begin{proof}
Let $x \in (A \cup B) \cap (A \cup C),$ show $x \in A \cup (B \cap C).$\\
By definitions of union and intersection, $x \in (A \cup B) \cap (A \cup C) \implies (x \in A \ \text{or} \ x \in B)$ and $(x \in A \ \text{or} \ x \in C).$ In $(x \in A \ \text{or} \ x \in B), x \notin A \implies x \in B$ (and similarly for $(x \in A \ \text{or} \ x \in C)$). Therefore, if $x \notin A,$ then $x \in B \cap C \implies x \in A \ \text{or} \ x \in B \cap C \implies x \in A \cup (B \cap C).$
\end{proof}
Since $A \cup (B \cap C) \subseteq (A \cup B) \cap (A \cup C)$ and $(A \cup B) \cap (A \cup C) \subseteq A \cup (B \cap C),$ we must have $=.$

\subsection*{$(xii) \enspace {^{\mathrm{c}}}(A \cap B) = \ {^{\mathrm{c}}}{A} \cup {^{\mathrm{c}}}{B} $}
\begin{lemma}
${^{\mathrm{c}}}(A \cap B) \subseteq \ {^{\mathrm{c}}}{A} \cup {^{\mathrm{c}}}{B} $
\end{lemma}
\begin{proof}
Let $x \in {^{\mathrm{c}}}(A \cap B),$ show $x \in {^{\mathrm{c}}}{A} \cup {^{\mathrm{c}}}{B}.$\\
By definition of complement,  $x \in {^{\mathrm{c}}}(A \cap B) \implies x \notin (A \cap B).$ For $x \notin (A \cap B),$ it must be true that $x \notin A$ or $x \notin B.$ Therefore, $x \in  {^{\mathrm{c}}}{A}$ or $x \in {^{\mathrm{c}}}{B} \implies x \in  {^{\mathrm{c}}}{A} \cup  {^{\mathrm{c}}}{B}.$
\end{proof}

\begin{lemma}
${^{\mathrm{c}}}{A} \cup {^{\mathrm{c}}}{B} \subseteq \ {^{\mathrm{c}}}(A \cap B)$
\end{lemma}
\begin{proof}
Let $x \in {^{\mathrm{c}}}{A} \cup {^{\mathrm{c}}}{B},$ show $x \in {^{\mathrm{c}}}(A \cap B).$\\
By definitions of complement and union, $x \in {^{\mathrm{c}}}{A} \cup {^{\mathrm{c}}}{B} \implies x \notin A$ or $x \notin B \implies x \notin (A \cap B) \implies x \in {^{\mathrm{c}}}(A \cap B).$
\end{proof}
Since ${^{\mathrm{c}}}(A \cap B) \subseteq \ {^{\mathrm{c}}}{A} \cup {^{\mathrm{c}}}{B} $ and ${^{\mathrm{c}}}{A} \cup {^{\mathrm{c}}}{B} \subseteq \ {^{\mathrm{c}}}(A \cap B)$, we must have $=.$

\section{Exercise 1.4.5.}
To see why $A \times \emptyset = \emptyset \times A = \emptyset$ is true, we must understand that the Cartesian product $A \times \emptyset$ is the set of all ordered pairs in which the first coordinate comes from set $A$ and the second coordinate comes from $\emptyset$. Because there are no objects in $\emptyset$, we do not have a coordinate. Thus, there are no ordered pairs in the Cartesian product $A \times \emptyset$ (and similarly for $\emptyset \times A).$ Because there are no elements in these Cartesian products, we have the empty set $\emptyset.$ Thus, $A \times \emptyset = \emptyset \times A = \emptyset.$

\section{Exercise 1.4.6.}
\begin{proof}
Choose any  $ (x,y)\in A\times B $, then according to definition of set multiplication $A \times B = \{ (a,b)\mid a \in A ,\space  b \in B \}$, therefore $x \in A$ and $y \in B$. Since A=B, then an element exists in A is a necessary and sufficient condition for an element to exist in B.  $x \in B$ and $y \in A$. Then, we can derive $ (x,y)\in B\times A $. Then we can see for any element that belongs to $A \times B$ belongs to $B \times A$, then according to equality of set, $A \times B = B \times A$.
\end{proof}

\section{Exercise 1.6.2.}

\begin{proof} Show that R is an equivalence relation.

\begin{definition}
A relation on a set is called an equivalence relation if it is reflexive, symmetric, and transitive.
We must show that the set is reflexive, symmetric, and transitive in order to prove that R is an equivalence relation. 
\end{definition}
For all a $\in X, (a, a) \in$ R. From this statement, it is true that a $\sim a$ for all values a. Therefore, R is reflexive. For a, b, c $\in X$, if (a, b) and (b, c) $\in R$, then c, a $\in R$.

From this statement, a $\sim b$ and b $\sim c$, therefore a $\sim c$. This demonstrates that R is transitive. We must now prove that R is symmetric in order to demonstrate that R is an equivalence relation.

Because a $\sim a$ and a $\sim b$, we must show that b $\sim a$ to demonstrate that R is symmetric. If c $\sim a$ and b $\sim c$, this implies that b $\sim a$. 

Therefore, R is symmetric in addition to being reflexive and transitive.

Thus, by satisfying Definition 4.1, R is a equivalence relation.
\end{proof}
\section{Exercise 1.6.5.}
\begin{proof}
Show that the relation is an equivalence relation.
Since a - a = 0, a $\sim$ a for all a $\in \mathbb{Z}$. Hence, this is reflexive.
As for symmetry:
\begin{center}
Suppose that $\frac{a-b}{k}$ where k $\in \mathbb{Z}$. 
Solve for k' such that k' $\in \mathbb{Z}$ and
$\frac{b-a}{n}$ = k'
\\
$\frac{b-a}{n}$ = $\frac{(a-b)}{n}$ = -k
\\
Since -k $\in \mathbb{Z}$, let k' = -k.
\\
\end{center}
Thus, the relation is symmetric.
\\
To show transitivity, 
\\
Suppose that $\frac{a-b}{k_1}$ and $\frac{b-c}{k_2}$, where $k_1, k_2$ $\in \mathbb{Z}$.
\begin{center}
    $\frac{a-c}{n}$ = $k_3$ where $k_3$ $\in \mathbb{Z}$
    $\frac{a-c}{n}$ = $\frac{a-b+b-c}{n}$ = $k_1$ + $k_2$
\end{center}
Since $k_1 + k_2$ $\in \mathbb{Z}$, $k_3 = k_1 + k_2 $
\\
Thus, the relation is transitive, and therefore an equivalence relation.
\end{proof}
\section{Exercise 1.6.10.}
%Aditya Krishna
If $c|a$, then we can say that $a=k_1c$, for some $k_1\in\mathbb{Z}$. If $c|b$, then we can also say that $b=k_2c$, for some $k_2\in\mathbb{Z}$. Then it follows that for some integers $s,t\in\mathbb{Z}$, we have $(sa+tb)=(sk_1c+tk_2c)$. Pulling out the $c$, we get $c(sk_1+tk_2)$. It follows then that $c|c(sk_1+tk_2)$, for some $sk_1+tk_2\in\mathbb{Z}$. As $(sk_1c+tk_2c) = (sa+tb)$ then we can see that $c|(sa+tb)$.

\section{Exercise 1.6.13.}
%Aditya Krishna
\begin{enumerate}
    \item $C(a) = \{b \in X | b \sim a\}$, and since $a\in X$, and $a\sim a$, $a$ must also be contained in the set $C(a)$, so we have $a\in C(a)$.
    \item 
    
    To show that $C(a)=C(b)$, we need $C(a)\subseteq C(b)$, and $C(b)\subseteq C(a)$.

    Let $C(a)=\{b_1\in X:b_1\sim a\}$, and $C(b)=\{b_2\in X:b_2\sim b\}$. Let $x$ be an arbitrary element of $C(a)$. Then $x\sim a$, and because $a\sim b$, we have $x\sim b$. So every element of $C(a)$ is contained within $C(b)$, or $C(a)\subseteq C(b)$. Let $y$ be an arbitrary element of $C(b)$. Then $y\sim b$, and because $b\sim a$, we have $y\sim a$. So every element of $C(b)$ is contained within $C(a)$, or $C(b)\subseteq C(a)$. So it follows that $C(a)=C(b)$.
        \item 
        \item $\bigcup_{a \in X} C(a) = X$. Because $a\in C(a)$, then $\{a\}\subseteq C(a)$. So we have $\bigcup\{a\}\subseteq\bigcup C(a)$, and the union of all $\{a\}$, for $a\in X$, will just be $X$. So $X\subseteq \bigcup C(a)$, but because $C(a)\subseteq X$, we have $\bigcup C(a)=X$. 


\end{enumerate}

\section{Exercise 1.6.15.}
\begin{proof}
To proof equivalence relation, three properties must be proved separately.
\\$\boldsymbol{ 1. reflexive \enspace property}$
\\ For any $(x,y) \in F$ , $x,y \neq 0$
\\ Since $x/y=x/y$,
therefore \{ (x,y),(x,y)\} $\in \enspace \sim$ 
\\  $\boldsymbol{ 2. symmetric \enspace property}$
\\  For any $(x1,y1), (x2,y2) \in F$ , $xi,yi \neq 0$
\\If \{ (x1,y1),(x2,y2)\} $\in \enspace \sim$ 
\\ then x1/y1 = x2/y2
\\ x2/y2=x1/y1 
\\Therefore, \{ (x2,y2),(x1,y1)\} $\in \enspace \sim$ 
\\  $\boldsymbol{ 3. transitive \enspace property}$
\\  For any $(x1,y1), (x2,y2),(x3,y3) \in F$ , $xi,yi \neq 0$
\\If \{ (x1,y1),(x2,y2)\} $\in \enspace \sim$ ,
 and\{ (x2,y2),(x3,y3)\} $\in \enspace \sim$ 
 \\then x1/y1=x2/y2, x2/y2=x3/y3
 \\then x1/y1=x3/y3
 \\Therefore, \{(x1,y1),(x3,y3)\} $\in \enspace \sim$ 
 \\
 \\since these three properties are met, $\sim $ operation is an equivalence relation. 

\end{proof}

\section{Non-Textbook Exercise}

Express each of the following statements as a conditional statement in "if-then" form. \\ \\

(a) Every odd number is prime.
\\Conditional: If a number is odd, then the number is prime.
\\Negation: There is an odd number that is not prime.
\\Converse: If a number is prime, then the number is odd.
\\Contrapositive: If a number is not prime, then the number is not odd. \\ \\

(b) Passing the test requires solving all the problems.
\\Conditional: If the test will be passed, then all the problems must be solved.
\\Negation: The test is passed without solving all the problems.
\\Converse: If all the problems are solved, then the test will be passed.
\\Contrapositive: If not all the problems are solved, then the test will not be passed. \\ \\

(c) Being first in line guarantees getting a good seat.
\\Conditional: If someone is first in line, then they will get a good seat.
\\Negation: Someone is first in line and does not get a good seat.
\\Converse: If someone gets a good seat, then they will have been first in line.
\\Contrapositive: If someone does not get a good seat, then they will not have been first in line. \\ \\

(d) I get mad whenever you do that.
\\Conditional: If you do that, then I will get mad. \\ \\

(e) I won't say that unless I mean it.
\\Conditional: If I say that, then I mean it.

\section{Exercise 1.7.9.}
\begin{proof}
To prove the pigeonhole principle, we take the negative proposition of the principle and then prove it wrong.
\\Therefore,If f:A$\to$ B is a function, there are only one distinct element of A that correspond to the same element of B.
\\Since f(a)=f(b) implies a=b for any a,b$\in$A, f is injective. 
\\Therefore $\abs A $ $\leq$ $\abs b$,  n$\leq$m.
\\However, in initial condition n$>$m.
\\The contradiction is found and the negative proposition is wrong. Therefore, the pigeonhole principle is valid. 


\end{proof}

\section{Exercise 1.7.15.}
For a given function f, determine if the f is surjective, injective, or bijective.

\subsection*{$(i) \enspace f: \mathbb{N} \to \mathbb{N} , f(n) = 2n $}
f(n) is not surjective.
\begin{definition}
For a given function $f(A) = \{ b \in B \ | \ \exists a \in A \ with \ f(a) = b \} $
\\ f is surjective if f(A) = B, where B is the range of f.
\end{definition}
For the given function f(n), 3 $\notin$ f(\mathbb{N}).
\\However, 3 \in \mathbb{N}, where \ \mathbb{N} \ is \ also \ the \ range \ of f(n).
\\Since \ 3 \in \mathbb{N} \ and \ 3 \notin f( \mathbb{N} ), \ \mathbb{N} \nsubseteq f( \mathbb{N} ). \ Therefore, \ \mathbb{N} \neq f( \mathbb{N} ).
\\f( \mathbb{N} ) \ is \ not \ equal \ to \ its \ range, \ and \ as \ such, \ this \ function \ is \ not \ surjective. \\ \\

f(n) is injective.
\begin{definition}
For a given function $f(A) = \{ b \in B \ | \ \exists a \in A \ with \ f(a) = b \} $
\\ f is injective if for any $a,a' \in A, if \ f(a) = f(a'), \ then \ a = a'$.
\end{definition}
\exists \ x,y \in $\mathbb{N}$
\\ Assuming \ f(x) = f(y)
\\ 2x = 2y
\\ x = y
\\Thus, \ f(n) \ is \ injective. \\ \\

f(n) is not bijective. If a function is bijective, then the function is surjective. f(n) is not surjective and thus is not bijective.

\subsection*{$(ii) \enspace f: \mathbb{Z} \to \mathbb{Z} , f(n) = n + 6 $}
f(n) is surjective.
\\ n \in \mathbb{Z},
\\ y = n + 6
\\ n = y - 6
\\ y \in \mathbb{Z}
\\ y = f(n), \ therefore \ f(\mathbb{Z}) \ equals \ the \ range \ \mathbb{Z}. \ Thus, \ f \ is \ a \ surjective \ function. \\ \\

f(n) is injective.
\\ \exists \ x,y \in \mathbb{Z}
\\ Assuming \ f(x) = f(y)
\\ x + 6 = y + 6
\\ x = y
\\ Thus, \ f(n) \ is \ injective. \\ \\

f(n) is bijective. A function is bijective if and only if the function is surjective and injective. f(n) is surjective and injective, thus f(n) is bijective.

\subsection*{$(iii) \enspace f: \mathbb{N} \to \mathbb{Q} , f(n) = n $}
f(n) is not surjective.
\\ \forall n \in \mathbb{N}
\\f(n) \in \mathbb{N}, \  f($\mathbb{N}$) \subseteq \mathbb{N}
\\ \frac{1}{3} \in \mathbb{Q}
\\ \frac{1}{3} \notin \mathbb{N} \implies \frac{1}{3} \notin f($\mathbb{N}$)
\\ If \ \frac{1}{3} \in \mathbb{Q} \ and \ \frac{1}{3} \notin f($\mathbb{N}$), \ then \ \mathbb{Q} \nsubseteq f($\mathbb{N}$).
\\ Therefore \ f($\mathbb{N}$) \neq \mathbb{Q}
\\ As \ such, \ f($\mathbb{N}$) \ does \ not \ equal\ the \ range \ \mathbb{Q}, \ therefore \ the \ function \ is \ not \ surjective. \\ \\

f(n) is not injective.
\\ \exists \ x,y \in \mathbb{N}
\\ Assuming \ f(x) = f(y)
\\ x = y
\\ Thus, \ f(n) \ is \ injective. \\ \\

f(n) is not bijective. If a function is bijective, then the function is surjective. f(n) is not surjective and thus is not bijective.

\subsection*{$(iv) \enspace f: \mathbb{Q}_{+} \to \mathbb{N} , f(\frac{a}{b}) = a_{n}...a_{2}a_{1}db_{m}...b_{2}b_{1} $}
Where $\mathbb{Q}_{+}$ is the set of rational numbers written as base 10 fractions and \mathbb{N} is the set of natural numbers but written in base 11. The output of f is one value whose digits are arranged in the above detailed order. \\ \\

f($\frac{a}{b}$) is not surjective. Since $\mathbb{Q}_{+}$ \ is \ always \ positive, \frac{a}{b} > 0.
\\ 1 \in \mathbb{N}
\\ 1 \notin \ f($\mathbb{Q}_{+}$)
(Since the smallest number f could yield would be 1d1 (a and b both must be non-0 and result in the form adb, if a or b are larger than 9 then the outcome will be in the base 11 thousands; the smallest possible outcome must use only single digit a,b values and thus will use the smallest value possible, 1), or 232 in base 10, 1 is not contained in the set of possible elements of $f{mathbb{Q}_{+}}$.)
\\ Therefore, \ \mathbb{N} \nsubseteq \ f($\mathbb{Q}_{+}$). \ f($\mathbb{Q}_{+}$) \neq \ \mathbb{N}.
\\ f($\mathbb{Q}_{+}$) \ is \ not \ equal \ to \ the \ range\ of \ the \ function \ and \ as \ such \ the \ function \ is \ not \ surjective. \\ \\

f($\frac{a}{b}$) is injective.
\\ \exists \frac{v}{w}, \frac{x}{y} \ \in $\mathbb{Q}_{+}$
\\ Assuming \ f($\frac{v}{w}$) = f($\frac{x}{y}$)
\\ vdw = xdy
\\ Since \ vdw = xdy, \ v = x, \ w = y, \ $\frac{v}{w}$= $\frac{x}{y}$.
\\ Therefore, \ f \ is \ injective. \\ \\

f($\frac{a}{b}$) is not bijective. If a function is bijective, then the function is surjective. f($\frac{a}{b}$) is not surjective and thus is not bijective.

\subsection*{$(v) \enspace f: \mathbb{R} \to \mathbb{N} , f(x) = a_{-3} $}
For the decimal expansion of any rational number $a_{n}...a_{2}a_{1}.a_{-1}a_{-2}a_{-3}a_{-4}...a_{m}$. \\ \\
f(x) is not surjective. $a_{-3}$ is contained in the set of digits $\{0,1,2,3,4,5,6,7,8,9\}$. Therefore, f($\mathbb{R}$)= $\{0,1,2,3,4,5,6,7,8,9\}$.
\\ 11 \in \mathbb{N}, \ 11 \notin \ f($\mathbb{R}$)
\\ Therefore, \ \mathbb{N} \nsubseteq f($\mathbb{R}$) \implies \mathbb{N} \neq f($\mathbb{R}$).
\\ f($\mathbb{R}$) \ does \ not \ equal \ the \ range \ of \ the \ function \ \mathbb{N}, \ therefore \ the \ function \ is \ not \ surjective. \\ \\

f(x) is not injective.
\\ \exists \ x = 1.001, y = 2.001 \in \mathbb{R}
\\ Assuming \ f(x) = f(y)
\\ f(1.001) = 1, f(2.001) = 1 \implies f(1.001) = f(2.001)
\\ 1.001 \neq 2.001
\\ x \neq y 
\\ Therefore, \ f(x) \ is \ not \ injective. \\ \\

f(x) is not bijective. If a function is bijective, then the function is surjective and injective. f(x) is neither surjective nor injective and thus is not bijective.

\end{document}