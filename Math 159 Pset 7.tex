

\documentclass[12pt]{article}


%----------Packages----------
\usepackage{amsmath}
\usepackage{amssymb}
\usepackage{amsthm}
%\usepackage{amsrefs}
\usepackage{dsfont}
\usepackage{mathrsfs}
\usepackage{stmaryrd}
\usepackage[all]{xy}
\usepackage[mathcal]{eucal}
\usepackage{verbatim}  %%includes comment environment
\usepackage{fullpage}  %%smaller margins
\usepackage{hyperref}
\usepackage{setspace}
\onehalfspacing
%----------Commands----------

%%penalizes orphans
\clubpenalty=9999
\widowpenalty=9999

%% bold math capitals
\newcommand{\bA}{\mathbf{A}}
\newcommand{\bB}{\mathbf{B}}
\newcommand{\bC}{\mathbf{C}}
\newcommand{\bD}{\mathbf{D}}
\newcommand{\bE}{\mathbf{E}}
\newcommand{\bF}{\mathbf{F}}
\newcommand{\bG}{\mathbf{G}}
\newcommand{\bH}{\mathbf{H}}
\newcommand{\bI}{\mathbf{I}}
\newcommand{\bJ}{\mathbf{J}}
\newcommand{\bK}{\mathbf{K}}
\newcommand{\bL}{\mathbf{L}}
\newcommand{\bM}{\mathbf{M}}
\newcommand{\bN}{\mathbf{N}}
\newcommand{\bO}{\mathbf{O}}
\newcommand{\bP}{\mathbf{P}}
\newcommand{\bQ}{\mathbf{Q}}
\newcommand{\bR}{\mathbf{R}}
\newcommand{\bS}{\mathbf{S}}
\newcommand{\bT}{\mathbf{T}}
\newcommand{\bU}{\mathbf{U}}
\newcommand{\bV}{\mathbf{V}}
\newcommand{\bW}{\mathbf{W}}
\newcommand{\bX}{\mathbf{X}}
\newcommand{\bY}{\mathbf{Y}}
\newcommand{\bZ}{\mathbf{Z}}

%% blackboard bold math capitals
\newcommand{\bbA}{\mathbb{A}}
\newcommand{\bbB}{\mathbb{B}}
\newcommand{\bbC}{\mathbb{C}}
\newcommand{\bbD}{\mathbb{D}}
\newcommand{\bbE}{\mathbb{E}}
\newcommand{\bbF}{\mathbb{F}}
\newcommand{\bbG}{\mathbb{G}}
\newcommand{\bbH}{\mathbb{H}}
\newcommand{\bbI}{\mathbb{I}}
\newcommand{\bbJ}{\mathbb{J}}
\newcommand{\bbK}{\mathbb{K}}
\newcommand{\bbL}{\mathbb{L}}
\newcommand{\bbM}{\mathbb{M}}
\newcommand{\bbN}{\mathbb{N}}
\newcommand{\bbO}{\mathbb{O}}
\newcommand{\bbP}{\mathbb{P}}
\newcommand{\bbQ}{\mathbb{Q}}
\newcommand{\bbR}{\mathbb{R}}
\newcommand{\bbS}{\mathbb{S}}
\newcommand{\bbT}{\mathbb{T}}
\newcommand{\bbU}{\mathbb{U}}
\newcommand{\bbV}{\mathbb{V}}
\newcommand{\bbW}{\mathbb{W}}
\newcommand{\bbX}{\mathbb{X}}
\newcommand{\bbY}{\mathbb{Y}}
\newcommand{\bbZ}{\mathbb{Z}}

%% script math capitals
\newcommand{\sA}{\mathscr{A}}
\newcommand{\sB}{\mathscr{B}}
\newcommand{\sC}{\mathscr{C}}
\newcommand{\sD}{\mathscr{D}}
\newcommand{\sE}{\mathscr{E}}
\newcommand{\sF}{\mathscr{F}}
\newcommand{\sG}{\mathscr{G}}
\newcommand{\sH}{\mathscr{H}}
\newcommand{\sI}{\mathscr{I}}
\newcommand{\sJ}{\mathscr{J}}
\newcommand{\sK}{\mathscr{K}}
\newcommand{\sL}{\mathscr{L}}
\newcommand{\sM}{\mathscr{M}}
\newcommand{\sN}{\mathscr{N}}
\newcommand{\sO}{\mathscr{O}}
\newcommand{\sP}{\mathscr{P}}
\newcommand{\sQ}{\mathscr{Q}}
\newcommand{\sR}{\mathscr{R}}
\newcommand{\sS}{\mathscr{S}}
\newcommand{\sT}{\mathscr{T}}
\newcommand{\sU}{\mathscr{U}}
\newcommand{\sV}{\mathscr{V}}
\newcommand{\sW}{\mathscr{W}}
\newcommand{\sX}{\mathscr{X}}
\newcommand{\sY}{\mathscr{Y}}
\newcommand{\sZ}{\mathscr{Z}}

\renewcommand{\phi}{\varphi}
%\renewcommand{\emptyset}{\O}

\providecommand{\abs}[1]{\lvert #1 \rvert}
\providecommand{\norm}[1]{\lVert #1 \rVert}
\providecommand{\x}{\times}
\providecommand{\ar}{\rightarrow}
\providecommand{\arr}{\longrightarrow}


%----------Theorems----------

\newtheorem{theorem}{Theorem}[section]
\newtheorem{proposition}[theorem]{Proposition}
\newtheorem{lemma}[theorem]{Lemma}
\newtheorem{corollary}[theorem]{Corollary}

\theoremstyle{definition}
\newtheorem{definition}[theorem]{Definition}
\newtheorem{nondefinition}[theorem]{Non-Definition}
\newtheorem{exercise}[theorem]{Exercise}

\numberwithin{equation}{subsection}



\begin{document}

\pagestyle{plain}


%%---  sheet number for theorem counter
%\setcounter{section}{1}   

\begin{center}
{\large Math 15910 Assignment 7} \\ 
{\medium Amatullah Mir} 
\vspace{.1 in}  
\end{center}
\section{Exercise 1}
\textbf{Use the Monotone Convergence Theorem to prove that if we define $e_{n}$ := ($1+1/n)^n$ then $(e_{n})$ converges. You may look up and use the Binomial Theorem.}

\begin{proof} We define the Monotone Convergence Theorem as follows.
\begin{definition}
If a series is bounded and monotone, then it converges.
\end{definition}
If we apply the Binomial Theorem:
\begin{center}
(1 + $\frac{1}{n})^n$ = 1 + 1 + $\frac{1-\frac{1}{n}}{2!}$} + $\frac{(1-\frac{1}{n})({1-\frac{2}{n}})}{3!}$}
\\
$\leq$ 1 + 1 + $\frac{1}{2}$ + $\frac{1}{6}$ + ...
\\
$<$ 1 + 1 + $\frac{1}{2}$ + $\frac{1}{4}$ + $\frac{1}{8}$ ... $\leq$ 3.

\end{center}
From this, we see that $e_{n}$ is increasing and bounded above by 3. It is increasing because each of $e_{n}$'s terms are smaller than the next terms when expanding $e_{n+1}$. Therefore, $e_{n}$ is convergent.
\end{proof}
\section{Exercise 2}
\textbf{If ($x_{n}$) is a convergent sequence, prove that it is Cauchy.}
\begin{proof} ($x_{n}$) is a convergent sequence. Let the limit of ($x_{n}$) = x.
\\
Let $\epsilon$ $>$ 0. There exists N $\in$ $\mathbb{N}$ such that for all n $\geq$ N,
\begin{center}
    $| x_{n} - x |$ $<$ $\epsilon/2$
\end{center}
By the triangle inequality, for all y, z $\geq$ $\mathbb{N}$:
\begin{center}
    $|x_{y} - x_{z}|$ $\leq$ $|x_{y}-x_{z}| + |x-x_{z}| < \epsilon/2 + \epsilon/2 =\epsilon$.
\end{center}
Thus, ($x_{n}$) is Cauchy.
\end{proof}
\section{Exercise 3}

\textbf{Show that if a Cauchy sequence does not converge to 0, all the terms must converge to the same sign.}

\begin{proof}
Let $x_{n}$ be a Cauchy sequence. 
\\\\Let $\epsilon > 0$. For all y, z $\geq$ N, $|x_{y} - x_{z}| < \epsilon$.
\\Since $x_{n}$ does not converge to 0, there exists $\epsilon > 0$ such that for each N,
$x_{n} \notin (0-\epsilon, 0+\epsilon)$.
\\\\
But for $\epsilon$, there exists a point N' such that for all other later points are within $\epsilon$ of one another, either less than -$\epsilon$ or less than $\epsilon$.
\\\\
Thus, since all later points past N' are less than -$\epsilon$ or less than $\epsilon$, all later points have to be either one sign, either all negative or all positive.
\end{proof}
\section{Exercise 4}

\textbf{Prove that the sequence
\begin{center} $x_{1} = 1$, $x_{n+1}$ = $\frac{1}{2}(x_{n} + \frac{3}{x_{n}})$
\end{center} is contracting. Compute its limit, and use calculator to compute $x_{5}$.}




\begin{proof}
Using the terms $x_{n+2}$ and $x_{n+1}$, we subtract:
\begin{center}
$|x_{n+2} - x_{n+1}| = |\frac{1}{2}(x_{n+1} + \frac{3}{x_{n+1}}) - \frac{1}{2}(x_{n} + \frac{3}{x_{n}})|$
\\
= $|\frac{1}{2}||x_{n+1} + \frac{3}{x_{n+1}} - (x_{n} + \frac{3}{x_{n}})|$
\\
= $|\frac{1}{2}||\frac{x^2_{n+1}+3}{x_{n+1}} + \frac{-x^2_{n}-3}{x_{n}}|$
\\
= $|\frac{1}{2}||\frac{(x^2_{n+1}x_{n}+3x_{n}-x^2_{n}x_{n+1}-3x_{n+1})}{x_{n+1}x_{n}}|$
\\
= $|\frac{1}{2}||\frac{-3(x_{n+1}-x_{n})}{x_{n+1}x_{n}}+\frac{x_{n+1}x_{n}(x_{n+1}-x_{n})}{x_{n+1}x_{n}}$
\\
= $|\frac{1}{2}(1-\frac{3}{x_{n+1}x_{n}})||x_{n+1}-x_{n}|$
\end{center}
If the sequence contracts, then the coefficient c will be less than 1, and $x_{n+1} $\approx$ x_{n}$ such that $|\frac{1}{2}(1-\frac{3}{x^2_{n}})|<1$.
\begin{center}
    $-2<1-\frac{3}{x^2_{n}}<2$ \\
    $-3<-\frac{3}{x^2_{n}}<1$ \\
    $1>\frac{1}{x^2_{n}}>-\frac{1}{3}$ \\ 
    $1>\frac{1}{x^2_{n}}$ \\ Solving for $x_{n},$ we get $x_{n}<-1$ or $x_{n}>1$.
\end{center}
Given that $x_{n}$ is a real number, there exists a number r $\in \mathbb{R}$ such that $x_{n} > r >1$. 
\\
Here, we obtain:
\\
= $|\frac{1}{2}(1-\frac{3}{x_{n+1}x_{n}})||x_{n+1}-x_{n}|<|\frac{1}{2}(1-\frac{3}{r^2})||x_{n+1}-x_{n}|<1|x_{n+1}-x_{n}|$
\\
Therefore, $|x_{n+2} - x_{n+1}| < c|x_{n+1}=x_{n}|$. \\ For c, the sequence is contracting (c=$|\frac{1}{2}(1-\frac{3}{r^2}|$.
r is the lower bound for $x_{n}$. To find r, we can prove that the value of r is a lower bound for all $x_{n}$ using induction.
For instance, because $x_{3} = \frac{7}{4} > \frac{3}{2}$, r may equal $\sqrt{\frac{3}{2}}$. 

\begin{center}
Given S(n) := $\sqrt{\frac{1}{2}} < x_{n}$, S(2) is our base case. S(2) = $\sqrt{\frac{1}{2}} < 2$. 
\\ We assume S(k) := $\sqrt{\frac{3}{2}}<x_{k}$, we prove that S(k+1) := $\sqrt{\frac{3}{2}}<x_{k+1}$
\\ $\frac{1}{2}(x_{k}+3_{x_k}) = x_{k+1}$. $x_{k} > \sqrt{\frac{3}{2}}$.
\\ Therefore, $x_{k+1}>\frac{1}{2}(\sqrt{\frac{3}{2}}+\frac{3}{\sqrt{\frac{3}{2}}})$ = $x_{k+1} > \frac{3}{2}\sqrt{\frac{3}{2}}>\sqrt{\frac{3}{2}}$\\
$x_{n} > \sqrt{\frac{3}{2}}$ for all n $\geq$ 2. 
\end{center}
\\Thus, $\sqrt{\frac{3}{2}}$ is a lower bound for $x_{n}$.
\\ When r = $\sqrt{\frac{3}{2}}$ and c = $|\frac{1}{2}(1-\frac{3}{\frac{3}{2}})|$, c = $|\frac{1}{2}(1-2)| = \frac{1}{2}(-1) = \frac{1}{2}$.
\\ As for the limit of the sequence, the limit is $\sqrt{3}$.
\\ The value for $x_{5} \approx 1.73205081$.
\end{proof}
\section{Exercise 5}

\textbf{Find the accumulation points of the following sets in $\mathbb{R}$:
\\ i) S = (0,1);
\\ ii) S = {$(-1)^{n}$ + $\frac{1}{n}$ $|$ n $\in$ $\mathbb{N}$};
\\ iii) S = $\mathbb{Q}$;
\\ iv) S = $\mathbb{Z}$;
}
\\
\\ i) [0,1]
\\ ii) \{-1,1\}
\\ iii) all points in $\mathbb{R}$.
\\ iv) no accumulation points.











\end{document}




