

\documentclass[12pt]{article}


%----------Packages----------
\usepackage{amsmath}
\usepackage{amssymb}
\usepackage{amsthm}
%\usepackage{amsrefs}
\usepackage{dsfont}
\usepackage{mathrsfs}
\usepackage{stmaryrd}
\usepackage[all]{xy}
\usepackage[mathcal]{eucal}
\usepackage{verbatim}  %%includes comment environment
\usepackage{fullpage}  %%smaller margins
\usepackage{hyperref}
\usepackage{setspace}
\onehalfspacing
%----------Commands----------

%%penalizes orphans
\clubpenalty=9999
\widowpenalty=9999

%% bold math capitals
\newcommand{\bA}{\mathbf{A}}
\newcommand{\bB}{\mathbf{B}}
\newcommand{\bC}{\mathbf{C}}
\newcommand{\bD}{\mathbf{D}}
\newcommand{\bE}{\mathbf{E}}
\newcommand{\bF}{\mathbf{F}}
\newcommand{\bG}{\mathbf{G}}
\newcommand{\bH}{\mathbf{H}}
\newcommand{\bI}{\mathbf{I}}
\newcommand{\bJ}{\mathbf{J}}
\newcommand{\bK}{\mathbf{K}}
\newcommand{\bL}{\mathbf{L}}
\newcommand{\bM}{\mathbf{M}}
\newcommand{\bN}{\mathbf{N}}
\newcommand{\bO}{\mathbf{O}}
\newcommand{\bP}{\mathbf{P}}
\newcommand{\bQ}{\mathbf{Q}}
\newcommand{\bR}{\mathbf{R}}
\newcommand{\bS}{\mathbf{S}}
\newcommand{\bT}{\mathbf{T}}
\newcommand{\bU}{\mathbf{U}}
\newcommand{\bV}{\mathbf{V}}
\newcommand{\bW}{\mathbf{W}}
\newcommand{\bX}{\mathbf{X}}
\newcommand{\bY}{\mathbf{Y}}
\newcommand{\bZ}{\mathbf{Z}}

%% blackboard bold math capitals
\newcommand{\bbA}{\mathbb{A}}
\newcommand{\bbB}{\mathbb{B}}
\newcommand{\bbC}{\mathbb{C}}
\newcommand{\bbD}{\mathbb{D}}
\newcommand{\bbE}{\mathbb{E}}
\newcommand{\bbF}{\mathbb{F}}
\newcommand{\bbG}{\mathbb{G}}
\newcommand{\bbH}{\mathbb{H}}
\newcommand{\bbI}{\mathbb{I}}
\newcommand{\bbJ}{\mathbb{J}}
\newcommand{\bbK}{\mathbb{K}}
\newcommand{\bbL}{\mathbb{L}}
\newcommand{\bbM}{\mathbb{M}}
\newcommand{\bbN}{\mathbb{N}}
\newcommand{\bbO}{\mathbb{O}}
\newcommand{\bbP}{\mathbb{P}}
\newcommand{\bbQ}{\mathbb{Q}}
\newcommand{\bbR}{\mathbb{R}}
\newcommand{\bbS}{\mathbb{S}}
\newcommand{\bbT}{\mathbb{T}}
\newcommand{\bbU}{\mathbb{U}}
\newcommand{\bbV}{\mathbb{V}}
\newcommand{\bbW}{\mathbb{W}}
\newcommand{\bbX}{\mathbb{X}}
\newcommand{\bbY}{\mathbb{Y}}
\newcommand{\bbZ}{\mathbb{Z}}

%% script math capitals
\newcommand{\sA}{\mathscr{A}}
\newcommand{\sB}{\mathscr{B}}
\newcommand{\sC}{\mathscr{C}}
\newcommand{\sD}{\mathscr{D}}
\newcommand{\sE}{\mathscr{E}}
\newcommand{\sF}{\mathscr{F}}
\newcommand{\sG}{\mathscr{G}}
\newcommand{\sH}{\mathscr{H}}
\newcommand{\sI}{\mathscr{I}}
\newcommand{\sJ}{\mathscr{J}}
\newcommand{\sK}{\mathscr{K}}
\newcommand{\sL}{\mathscr{L}}
\newcommand{\sM}{\mathscr{M}}
\newcommand{\sN}{\mathscr{N}}
\newcommand{\sO}{\mathscr{O}}
\newcommand{\sP}{\mathscr{P}}
\newcommand{\sQ}{\mathscr{Q}}
\newcommand{\sR}{\mathscr{R}}
\newcommand{\sS}{\mathscr{S}}
\newcommand{\sT}{\mathscr{T}}
\newcommand{\sU}{\mathscr{U}}
\newcommand{\sV}{\mathscr{V}}
\newcommand{\sW}{\mathscr{W}}
\newcommand{\sX}{\mathscr{X}}
\newcommand{\sY}{\mathscr{Y}}
\newcommand{\sZ}{\mathscr{Z}}

\renewcommand{\phi}{\varphi}
%\renewcommand{\emptyset}{\O}

\providecommand{\abs}[1]{\lvert #1 \rvert}
\providecommand{\norm}[1]{\lVert #1 \rVert}
\providecommand{\x}{\times}
\providecommand{\ar}{\rightarrow}
\providecommand{\arr}{\longrightarrow}


%----------Theorems----------

\newtheorem{theorem}{Theorem}[section]
\newtheorem{proposition}[theorem]{Proposition}
\newtheorem{lemma}[theorem]{Lemma}
\newtheorem{corollary}[theorem]{Corollary}

\theoremstyle{definition}
\newtheorem{definition}[theorem]{Definition}
\newtheorem{nondefinition}[theorem]{Non-Definition}
\newtheorem{exercise}[theorem]{Exercise}

\numberwithin{equation}{subsection}



\begin{document}

\pagestyle{plain}


%%---  sheet number for theorem counter
%\setcounter{section}{1}   

\begin{center}
{\large Math 15910 Assignment 6} \\ 
{\medium Amatullah Mir} 

\vspace{.2in}  
\end{center}



\section{Exercise}
\textbf{Prove that ($x_{n}$) converges if and only if its m-tail ($x_{m}$) converges. Moreover, show that they converge to the same limit. Think about how this strengthens many of our convergence results.}

\begin{proof}
Let ($x_{n}$) be a sequence that converges to L.
Let ($x_{m}$) be a sequence that converges to L such that 
\begin{center}
    ($x_{m}$) = ($x_{m+n}$) = ($x_{m+1}, x_{m+2}, x_{m+3}...$) 
\end{center}
Let $\epsilon > 0$. \\
Assume that m is the m-th tail of the sequence that is greater than or equal to 1. \\
($x_{n+m-1}$) = ($x_{n}$). \\
There exists N $\in \mathbb{N}$ such that
\begin{center}
$|(x_{n} - L)| \leq \epsilon$ when n $\geq$ N.
\end{center}

If n $\geq$ N, then $n + m - 1 \geq n \geq N.$
\\
$|(x_{m} - L)| = |(x_{n + m - 1} - L)| < \epsilon$ when n $\geq$ N, because $n + m - 1 \geq N$.
\\
\\
Because it is an if and only statement, we must prove the converse.
\\
$N_{0}$ is chosen so for all n $\geq N_{0}$:
\begin{center}
    $N = N_{0} + i - 1$.
\end{center}
If $n \geq N$, $n - i + 1 \geq N_{0}$.
\\
Also, $N_{0} \geq 1$ and $N = N_{0} + i - 1$, for any $n \geq N$, $n \geq N_{0} + i - 1$.
This implies that $n \geq N$, $n - i + 1 \geq N_{0} \geq 1$.
\\
Thus:
\begin{center}
    $| (x_{m-n+1} - L) | = | (x_{n} - L)|$
\end{center}
\end{proof}

\textbf{In general, this proof helps us  look at convergence differently by allowing us to look at sub-sequences and how they influence convergences of a sequence.}













\end{document}




