% Aditya Krishna, Amatullah Mir, Edwin Santos, and Zixuan Zhang (Group 6)


\documentclass[12pt]{article}


%----------Packages----------
\usepackage{amsmath}
\usepackage{amssymb}
\usepackage{amsthm}
%\usepackage{amsrefs}
\usepackage{dsfont}
\usepackage{mathrsfs}
\usepackage{stmaryrd}
\usepackage[all]{xy}
\usepackage[mathcal]{eucal}
\usepackage{verbatim}  %%includes comment environment
\usepackage{fullpage}  %%smaller margins
\usepackage{hyperref}
\usepackage{setspace}
\onehalfspacing
%----------Commands----------

%%penalizes orphans
\clubpenalty=9999
\widowpenalty=9999

%% bold math capitals
\newcommand{\bA}{\mathbf{A}}
\newcommand{\bB}{\mathbf{B}}
\newcommand{\bC}{\mathbf{C}}
\newcommand{\bD}{\mathbf{D}}
\newcommand{\bE}{\mathbf{E}}
\newcommand{\bF}{\mathbf{F}}
\newcommand{\bG}{\mathbf{G}}
\newcommand{\bH}{\mathbf{H}}
\newcommand{\bI}{\mathbf{I}}
\newcommand{\bJ}{\mathbf{J}}
\newcommand{\bK}{\mathbf{K}}
\newcommand{\bL}{\mathbf{L}}
\newcommand{\bM}{\mathbf{M}}
\newcommand{\bN}{\mathbf{N}}
\newcommand{\bO}{\mathbf{O}}
\newcommand{\bP}{\mathbf{P}}
\newcommand{\bQ}{\mathbf{Q}}
\newcommand{\bR}{\mathbf{R}}
\newcommand{\bS}{\mathbf{S}}
\newcommand{\bT}{\mathbf{T}}
\newcommand{\bU}{\mathbf{U}}
\newcommand{\bV}{\mathbf{V}}
\newcommand{\bW}{\mathbf{W}}
\newcommand{\bX}{\mathbf{X}}
\newcommand{\bY}{\mathbf{Y}}
\newcommand{\bZ}{\mathbf{Z}}

%% blackboard bold math capitals
\newcommand{\bbA}{\mathbb{A}}
\newcommand{\bbB}{\mathbb{B}}
\newcommand{\bbC}{\mathbb{C}}
\newcommand{\bbD}{\mathbb{D}}
\newcommand{\bbE}{\mathbb{E}}
\newcommand{\bbF}{\mathbb{F}}
\newcommand{\bbG}{\mathbb{G}}
\newcommand{\bbH}{\mathbb{H}}
\newcommand{\bbI}{\mathbb{I}}
\newcommand{\bbJ}{\mathbb{J}}
\newcommand{\bbK}{\mathbb{K}}
\newcommand{\bbL}{\mathbb{L}}
\newcommand{\bbM}{\mathbb{M}}
\newcommand{\bbN}{\mathbb{N}}
\newcommand{\bbO}{\mathbb{O}}
\newcommand{\bbP}{\mathbb{P}}
\newcommand{\bbQ}{\mathbb{Q}}
\newcommand{\bbR}{\mathbb{R}}
\newcommand{\bbS}{\mathbb{S}}
\newcommand{\bbT}{\mathbb{T}}
\newcommand{\bbU}{\mathbb{U}}
\newcommand{\bbV}{\mathbb{V}}
\newcommand{\bbW}{\mathbb{W}}
\newcommand{\bbX}{\mathbb{X}}
\newcommand{\bbY}{\mathbb{Y}}
\newcommand{\bbZ}{\mathbb{Z}}

%% script math capitals
\newcommand{\sA}{\mathscr{A}}
\newcommand{\sB}{\mathscr{B}}
\newcommand{\sC}{\mathscr{C}}
\newcommand{\sD}{\mathscr{D}}
\newcommand{\sE}{\mathscr{E}}
\newcommand{\sF}{\mathscr{F}}
\newcommand{\sG}{\mathscr{G}}
\newcommand{\sH}{\mathscr{H}}
\newcommand{\sI}{\mathscr{I}}
\newcommand{\sJ}{\mathscr{J}}
\newcommand{\sK}{\mathscr{K}}
\newcommand{\sL}{\mathscr{L}}
\newcommand{\sM}{\mathscr{M}}
\newcommand{\sN}{\mathscr{N}}
\newcommand{\sO}{\mathscr{O}}
\newcommand{\sP}{\mathscr{P}}
\newcommand{\sQ}{\mathscr{Q}}
\newcommand{\sR}{\mathscr{R}}
\newcommand{\sS}{\mathscr{S}}
\newcommand{\sT}{\mathscr{T}}
\newcommand{\sU}{\mathscr{U}}
\newcommand{\sV}{\mathscr{V}}
\newcommand{\sW}{\mathscr{W}}
\newcommand{\sX}{\mathscr{X}}
\newcommand{\sY}{\mathscr{Y}}
\newcommand{\sZ}{\mathscr{Z}}

\renewcommand{\phi}{\varphi}
%\renewcommand{\emptyset}{\O}

\providecommand{\abs}[1]{\lvert #1 \rvert}
\providecommand{\norm}[1]{\lVert #1 \rVert}
\providecommand{\x}{\times}
\providecommand{\ar}{\rightarrow}
\providecommand{\arr}{\longrightarrow}


%----------Theorems----------

\newtheorem{theorem}{Theorem}[section]
\newtheorem{proposition}[theorem]{Proposition}
\newtheorem{lemma}[theorem]{Lemma}
\newtheorem{corollary}[theorem]{Corollary}

\theoremstyle{definition}
\newtheorem{definition}[theorem]{Definition}
\newtheorem{nondefinition}[theorem]{Non-Definition}
\newtheorem{exercise}[theorem]{Exercise}

\numberwithin{equation}{subsection}



\begin{document}

\pagestyle{plain}


%%---  sheet number for theorem counter
%\setcounter{section}{1}   

\begin{center}
{\large Math 15910 Assignment 5} \\ 
\vspace{.2in}  
\end{center}



\section{Exercise 1}
Show that, for any $a, \ b \in \mathbb{Q}, \ we \ have \ \abs{\abs{a}-\abs{b}} \leq \abs{a - b}$

\begin{proof}
$\abs{\abs{a}-\abs{b}} \leq \abs{a - b}$

\\ $\abs{\abs{a}-\abs{b}}$ = $\abs{a}$ - $\abs{b}$ if $\abs{a} \geq \abs{b}$ = $\abs{b}$ - $\abs{a}$ if $\abs{b} \geq \abs{a}$ 
\\ = $\abs{a}$ + (-$\abs{b}$) if $\abs{a} \geq \abs{b}$ = $\abs{b}$ + (-$\abs{a}$) if $\abs{b} \geq \abs{a}$ 
\\ ( $\abs{-a} = \abs{a}$ = a, $\abs{b} = \abs{-b}$ = b $\implies$)
\\ = a + (-b) if $\abs{a} \geq \abs{b}$ = b + (-a) if $\abs{b} \geq \abs{a}$ \\ \\

$\abs{a - b}$ = a - b if a $\geq$ b = b - a if b $\geq$ a
\\ = a + (-b) if a $\geq$ b = b + (-a) if b $\geq$ a
\\ = a + (-b) if a $\geq$ b and b > 0 = b + (-a) if b $\geq$ a and a > 0

\\ $\implies$ $\abs{\abs{a}-\abs{b}}$ = a + (-b) = $\abs{a - b}$ (if a $\geq$ b and b > 0) \\ \\

$\abs{a - b}$ = a + (-(-b)) if a $\geq$ b and b < 0 = b + (-(-a)) if b $\geq$ a and a < 0
\\ = a + b if a $\geq$ b and b < 0 = b + a if b $\geq$ a and a < 0 \\ \\

If -b < 0 < b, then a + (-b) < a < a + b and a + (-b) < a + b
\\ $\implies$ (if a $\geq$ b and b > 0) $\abs{\abs{a}-\abs{b}}$ = $\abs{a - b}$ = a + (-b) < a + b  = $\abs{a - b}$ (if a $\geq$ b and b < 0)

\\ $\abs{\abs{a}-\abs{b}}$ = $\abs{a - b}$ < $\abs{a - b}$ $\implies$ $\abs{\abs{a}-\abs{b}}$ $\leq$ $\abs{a - b}$

\end{proof}
\section{Exercise 2}

If we remove the assumption that the intervals are closed then there are cases where there is no LUB (as there can not be a least element in T that is greater than or equal to all elements of S assuming S is a subset of a partially ordered set T) which would cause $A$ to not be bounded above and therefore a would not necessarily be an element of $\cap_n\epsilon\mathbb{N} [a_n,b_n]$.
\\
If the intervals are not nested then $[a_n+1,b_n+1]$ is not necessarily a subset of $[a_n,b_n]$ which means that there are cases where the $a \epsilon [a_n+1,b_n+1]$ but $a \notin [a_n,b_n]$ which would make there be cases where a would not necessarily be an element of $\cap_n\epsilon\mathbb{N} [a_n,b_n]$.

\section{Exercise 3}
\textbf{Let $K_{n}$ = (n, $\infty$) for all n $\in \mathbb{N}$.  Prove that $\cap_{n \in \mathbb{N}}$ $K_{n} = $\emptyset$}$
\begin{proof} 
Proof by contradiction. 
\\ Assume there exists a number n $\in$ $\cap_{n \in \mathbb{N}}K_{n}$.
\\
By definition of the Archimedean Property, there exists a natural number $n_{0}$ in which x $<$ $n_{0}$. 
However, x $\notin K_{n_{0}}$.
\\
This contradicts the initial number n $\in \cap_{n \in \mathbb{N}}K_{n}$. Therefore, there is no real number n $\in$ $\cap_{n \in \mathbb{N}}K_{n}$.
Thus, we prove that $\cap_{n \in \mathbb{N}}K_{n}$= $\emptyset$.
\end{proof}
\section{Exercise 4}


Claim $x \in I_n \forall n \in \mathbb{N}$. By Theorem 3.44 $A$ is bounded above by $b_1$. This means that $b_1$ is the supremum of the interval $I_n$. We can see that $x$ would be the supremum of A because $A \leq b_n$. $x$ would be unique as any $y$ contained in the intersection would have to be equal to $x$ as if $y$ were less than $x$, $x$ would not be the supremum and if it were greater than $x$ then it would be greater than $b_n$ and therefore not in the interval $I_n$.

\section{Exercise 5}
\textbf{Show that $n^3 - 6 \geq n^2$ for all sufficiently large n $\in \mathbb{N}$.}
\\
\\
There exists P = \{$n^3 - 6 \geq n^2$\}
We use mathematical induction to prove this statement is true.
If n = 5,
\begin{center}
    $(5)^3 - 6 \geq (5)^2$ $\implies$ $125 - 6 \geq 25$ $\implies$ $ 119 \geq 25. $ P is true for n = 5.
\end{center}
Now, we assume P is true for n = k.
\begin{center}
    $(k)^3 - 6 \geq (k)^2$ 
\end{center}
Now, we must prove that P is true for n = k + 1. We add $3k^2+3k+1$ to $(k)^3 - 6 \geq (k)^2$.
\begin{center}
    $k^3 - 6 + 3k^2+3k+1 \geq k^2 + 3k^2+3k+1$.
    \\
    $k^3+3k^2+3k+1-6 \geq k^2 + 2k+1+(3k^2+k)$
    \\
    $(k+1)^3-6 \geq (k+1)^2 + 3k^2+k$
    \\
    $(k+1)^3-6 \geq (k+1)^2$
\end{center}
Therefore, P is true for n = k + 1. 
\\
Since P is true for n = 5, n = k, and n = k + 1, $n^3 - 6 \geq n^2$ is true for all sufficiently large n $\in \mathbb{N}$.
\section{Exercise 6}
To prove convergence, we prove that the sequence converge to 4. 
\\Then we have $|(4n^3+3n)/(n^3-6)-4|=|(3n+24)/(n^3-6)|$
\\Since $n^3 - 6 \geq n^2$ for large enough $n \geq M$.
\\$|(3n+24)/(n^3-6)| \leqslant (3n+24)/n^2 \leqslant 3/n+24/n = 27/n $
\\ then adopting Archimedian Property we have integer N that satisfies $1/N < E/27 $ for every positive real E. Then we have for $n \geq max(N,M)$, we have $|(4n^3+3n)/(n^3-6)-4|=|(3n+24)/(n^3-6)| < E$.
\\We prove that $(4n^3+3n)/(n^3-6)$ converges.   
\section{Exercise 7}
To prove convergence, we prove that the sequence converge to 6.
\\Then we have $|(12n^5+3n^4)/(2n^5-1)-6|=(3n^4+6)/(2n^5-1) \leqslant (3n^4-1.5)/(2n^5-n) + 7.5/(2n^5-1)$
\\Then we prove $2n^5-1 \geq n^4$. Since $2n^5-n^4-1=n^4(2n-1-1/n^4)$ and $1/n^4 \leqslant 1$, then we have  $2n-1-1/n^4 \geq 2n-2$. Therefore for all n in positive integer, the $2n^5-1 \geq n^4$. 
\\ Then we use the proved statement and acquire $(3n^4-1.5)/(2n^5-n) + 7.5/(2n^5-1) \geq 1.5/n +7.5/n^4 \geq 9/n $,
\\We can finally use the Archimedian property. Exist an integer N, $1/N < E/9 $ for every positive real E.
Then we have for n bigger than N, we can always have $9/n < E$ and then $|(12n^5+3n^4)/(2n^5-1)-6| < E$. The sequence converges.  
\section{Exercise 8}

If $s = 0$, this means that $s_n \to 0$. Since $s_n \to 0$, there exists a positive integer $n$ such that $s_n < \epsilon$. Because $s_n$ is a strictly positive sequence and it is decreasing $\sqrt{s_n} \leq s_n$. Then for all $n \in \mathbb{N}$ 
\\
$\sqrt{s_n} \leq s_n$ and as $s_n < \epsilon$ so $\sqrt{s_n} \leq \epsilon$. Therefore since $s_n \to 0$ $\sqrt{s_n} \to 0$.
\\
For $s > 0$ the same concepts apply and since $s_n$ is strictly positive $\sqrt{s_n} \to \sqrt{s}$ as there is no negative numbers and therefore $s \in \mathbb{R}$. 

\section{Exercise 9}
Prove that $(x_n)_{n\in \mathbb(N)}$ converges to 0 if and only if $(\abs{x_n})_{n\in \mathbb(N)}$ converges to 0.

\begin{proof}
Theorem: Let $(a_n) \to$ 0 with with $a_n \geq$ 0 and x $\in \mathbb{R}$.
\\ If $\exists$ c $>$ 0 and $\exists$ M $\in \mathbb{N}$ such that $\abs{x_n - x} \leq C*a_n \ \forall n \geq$ M, then ($x_n$) $\to$ x. \\ \\

If x is 0 and thus $(x_n)_{n\in \mathbb(N)}$ converges to 0, then $\abs{x_n - 0} \leq C*a_n$.
Therefore, $\abs{x_n} \leq C*a_n$
Since C*$a_n$ converges to 0, $\abs{x_n} \forall n \geq$ M is $\leq 0$, such that $\abs{x_n}$ must converge to 0.

$\therefore$ When $(x_n)_{n\in \mathbb(N)}$ converges to 0, $(\abs{x_n})_{n\in \mathbb(N)}$ will also always converge to 0.

\end{proof}

Give an example showing that converge of $(\abs{x_n})_{n\in \mathbb(N)}$ in general does not imply convergence of $(x_n)_{n\in \mathbb(N)}$.

\begin{proof}
$(\abs{x_n})_{n\in \mathbb{N}} = \abs{(-1)^n(1 + \frac{1}{n})}$ converges (to 1).
\\ $(x_n)_{n \in \mathbb{N}} = (-1)^n(1 + \frac{1}{n})$ diverges.
\\ When n is even, $(x_n)_{n \in \mathbb{N}}$ = $((-1)^2)^n(1 + \frac{1}{n})$ = $1^n*(1 + \frac{1}{n})$ which converges to 1.
\\ When n is odd, $(x_n)_{n \in \mathbb{N}}$ = $(-1)^{2n+1}(1 + \frac{1}{n})$ = $(-1)^{2n}*(-1)^{1}*(1 + \frac{1}{n})$ = $1*-1*(1 + \frac{1}{n})$ which converges to -1.
\\ $\therefore$ The convergence of $(\abs{x_n})_{n\in \mathbb(N)}$ does not imply the convergence of $(x_n)_{n \in \mathbb{N}}$.
\end{proof}

\end{document}